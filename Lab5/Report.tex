\documentclass[12pt]{report}

\usepackage{amsmath,amssymb,amsfonts}
\usepackage{listings}
\usepackage{graphicx}
\usepackage{hyperref}
\usepackage{fancyhdr}
\usepackage{color}
\usepackage{lettrine}
\usepackage{textcomp}
\usepackage[utf8]{inputenc}
\usepackage[toc,page]{appendix}
\usepackage{courier}

\usepackage[margin=2cm]{geometry}

\definecolor{dgrey}{rgb}{0.3,0.3,0.3}
\definecolor{dyellow}{rgb}{0.6,0.6,0.0}
\definecolor{gblue}{rgb}{0.35,0.35,0.6}

\hypersetup{
	colorlinks=true,
	linktoc=all,
	linkcolor=gblue,
	citecolor=dyellow,
}

\lstset{basicstyle=\ttfamily\footnotesize, frame=single,
breaklines=true,showstringspaces=false,numbers=left}

\title{ECSE 323 - Digital System Design\\After Action Report - The Craziest Eights Known To Man}
\author{Harley Wiltzer (260690006)\\Spiros-Daniel Mavroidakos (260689391)}
\date{April 11, 2017}

\begin{document}
\pagenumbering{gobble}
\maketitle
\newpage
\pagestyle{fancy}
\fancyhf{}
\rhead{Harley Wiltzer (260690006), Spiros-Daniel Mavroidakos (260689391) - Vandelay Industries}
\pagenumbering{arabic}
\tableofcontents
\part[A Barbarous Prolegomenon]{A Barbarous Prolegomenon\\\begin{center}\textnormal{\normalsize
		\\\textit{``Every hand's a winner,\\And every hand's a loser,\\So the best that you can hope
		for\\Is to die in your sleep''\\ - Kenny Rogers, The Gambler}}\end{center}}
\chapter*{A Barbarous Prolegomenon}
\lettrine{H}ere we are at last: after many weeks of frustration, anger, euphoria, and fine
documentation, the crazy eights system has finally experienced completion. The familiar reader
hopefully learned not only about digital systems throughout this daring adventure, but also about
love, about life, and about the categorical imperative. The reader is strongly discouraged,
regardless of his exposure to the previous reports that have been published, to avoid ignoring this
final frontier, as not only will it open the doors to unconquered kinds of clarity, but it will clear up
and fill the \textit{lacunae} of the previous readings as well.\\\\
There was once a time when neither Wiltzer or Mavroidakos could write a single line of VHDL without
saying the Lord's name in vain. On that note, there was also a time when neither Wiltzer or
Mavroidakos could tie their shoes without assistance. But was there ever a time when both Wiltzer
and Mavroidakos \textit{had no shoes}? The point that is being made here is that while Wiltzer and
Mavroidakos may, at one point, have not been able to design VHDL descriptions with any marginal
degree of competence, that \textit{is not} an indication that Wiltzer and Mavroidakos will never be
competent enough to write VHDL. This is because although there was a time when Wiltzer and
Mavroidakos could not write respectable VHDL, they always \textit{had} VHDL in their souls. In fact,
not only will this lab report demonstrate a considerable
degree of competence in VHDL code, but the VHDL written by Wiltzer and Mavroidakos is projected by
some to be considered VHDL's ``The Catcher in the Rye'' by February 2024.\\\\
Since the conception of the infamous Bi-directional Jenga Tower (known as a stack to the
less-informed) of Laboratory Session 3, it has been at the peak of digital systems enthusiasts'
focus. However, it was not good enough for its developers, as it was deemed ``sloppy'', ``bloated'',
and ``slightly cancerous''. As such, a new and improved Bi-directional Jenga Tower was designed for
the implementation of the full Crazy Eights system, and it was done using VHDL alone. Of exceptional
note here is that the BJT was designed without Quartus. No hot, sticky, messy Quartus, no Quartus at
all! Should this be considered ingenious, or should it be considered humanity's greatest
mistake?\\\\
Allow me to be Frank: There were several occasions throughout the writings of the past four
laboratory reports that exhibited some mild sarcasm. Of particular note was the constant praise of
the Altera Quartus II and Modelsim software, which are indeed so inconvenient that not even
Mother Theresa would save them from a fire. In fact, the horrendous user experience provided by
Quartus II and Modelsim are what \textit{prompted} the designers to create the new and improved BJT
in VHDL. To answer the concluding question of the previous paragraph, the decision to avoid Quartus
in the design of the new BJT has so far been considered ingenious by even the most qualified of
bystanders. As the attentive reader may pick up on throughout this report, one of the incredible
features of the new BJT is that it's depth is modular - that is to say, the same circuit block can
be used to create a 52-card stack and a 26-card stack. This ultimately made the design of the full
system far cleaner and easier to manage.\\\\
It is not the author's intention to spoil the remainder of this report by giving away all of the
juicy details in this barbarous prolegomenon. Therefore, it is intended to conclude the beginning of
this report prophylactically before it is no longer the beginning. But first, the reader should be
warned that what
is about to be exhibited is \textit{particularly} intense. For this reason, for the optimal experience,
it is suggested that the reader digest this report with lots of paper handy and a fountain pen in
hand, with tissue paper available, and in close vicinity to one or more restrooms. \\\\
Furthermore, it
is important to understand that this report \textit{is not} a novel, and thus should not be read
like one. Do not attempt to read this report like you would read ``Cloudy with a Chance of
Meatballs'', for example. Instead, this report should be read in a similar fashion to how one might
read ``The Origin of Species'', which is considered by many to be on approximately the same scale in
terms of intellectual \textit{étalage} and scientific significance.\\\\
Finally, although communication by writing is not done face to face, the authors feel that this
report may be the analogue to the last time they'll ever see you again. Special care was taken to
ensure that this report was entirely complete, so that clarity may be reliably established.
Although it is unfortunate that this channel of communication must cease after the publishing of
this report, do not interpret this as a sad or even moderately-unhappy occasion. In truth, a lot was
learned over the course of these laboratory sessions, and such education was made possible by the
support of the readers. Although this dialog must end, the lessons, discoveries, and memories of
these laboratory sessions will be around forever. \\\\That's what writing is for, after all. 
\end{document}
