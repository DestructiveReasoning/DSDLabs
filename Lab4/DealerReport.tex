\documentclass[12pt]{report}

\usepackage{amsmath,amssymb,amsfonts}
\usepackage{listings}
\usepackage{graphicx}
\usepackage{hyperref}
\usepackage{fancyhdr}
\usepackage{color}
\usepackage{lettrine}
\usepackage{textcomp}
\usepackage[utf8]{inputenc}
\usepackage[toc,page]{appendix}
\usepackage{courier}

\usepackage[margin=2cm]{geometry}

\definecolor{dgrey}{rgb}{0.3,0.3,0.3}
\definecolor{dyellow}{rgb}{0.6,0.6,0.0}
\definecolor{gblue}{rgb}{0.35,0.35,0.6}

\hypersetup{
	colorlinks=true,
	linktoc=all,
	linkcolor=gblue,
	citecolor=dyellow,
}

\lstset{basicstyle=\ttfamily\footnotesize, frame=single,
breaklines=true,showstringspaces=false,numbers=left}

\title{ECSE 323 - Digital System Design\\After Action Report - The Dealer Finite State Machine}
\author{Harley Wiltzer (260690006)\\Spiros-Daniel Mavroidakos (260689391)}
\date{March 30, 2017}

\begin{document}
\pagenumbering{gobble}
\maketitle
\newpage
\pagestyle{fancy}
\fancyhf{}
\rhead{Harley Wiltzer (260690006), Spiros-Daniel Mavroidakos (260689391) - Vandelay Industries}
\pagenumbering{arabic}
\tableofcontents

\part{A Pleasant Preamble}
\label{s:preamble}
\lettrine{S}ometimes in life one may be presented with situations that make him rethink his beliefs and
question who, in fact, he really is. It is moments like these that actually define an individual
\textit{ad postremum}. Of course, it is up to the individual in question to \textit{realize} that he shall be
withdrawing cathexis from the myriad objects of empirical reality around him if enlightenment should
be obtained. People that occasionally experience this enlightenment are called patricians. Those who
continuously experience this enlightenment are called Engineers.\\\\
Consider the infamous game of Blackjack. Some may enjoy this game as a nice way to pass the time,
others may be violently obsessed with it. Regardless of who is playing, the game can only be played
reliably when players understand and follow the rules and when a credible dealer is present.
Naturally, the players are needed for the game to be played, and of course the game is only really
\textit{being played} when the rules are followed. The presence of the dealer, however, is far more
interesting than what the uninformed reader may believe.\\\\
Some say the dealer's job is to deal the cards, but that is a vast and decadent oversimplification.
Sure, the dealer \textit{should} deal the cards (at the appropriate times, that is), but the dealer
is also responsible for establishing structure and ensuring that the game does not get out of hand.
In fact, the rules of the game themselves have their strings pulled by the dealer. \\But some dealers
do not follow the rules.\\\\
It is no longer news that the goal of these laboratory sessions is to build a \textit{crazy eights}
game, and not a Blackjack game. Although the rules of the two games are different, both take on the
rules-dealer paradigm of gameplay. The layman, and even the patrician, may focus on the rules of the
game principally. However, as Engineers, the main focus of this laboratory was to create not just a
functional dealer, but a reliable, ethical, and ultimately compliant dealer. As shown later in this
report, this was successfully accomplished due to the design of a clever \textit{finite state
machine}, a construct that has also been used in the design of underwhelming robots.\\\\
In the interest of moral behavior, it would be unethical to display the accomplishments of this
laboratory session without giving due credit to the Altera Quartus II and Modelsim software, whose
magical functionalities allow such complex designs to be mapped onto an FPGA. Given Altera's great
text editing accomodations and incredible timing simulation tools, the development of this system
was dream-like.\\\\
At this stage, the reader is invited to explore the remainder of this report, which will go through
in a (hopefully) simple and organized manner the discoveries made during this laboratory session.
Beyond that, this report will discuss \textit{how} these discoveries were made, and how they were
reinforced, so the reader may gain intuition on developing state of the art digital systems.\\\\
Please enjoy the discoveries and details that follow, and try to learn something from them. Much is
to be gained
by grasping the concepts of the rules-dealer paradigm, as they apply to more in life than simply
digital systems and card games. To conclude this pleasant preamble, the reader is encouraged to,
above all, \textit{have fun} with this report, and better yet, to have fun with life. Finally, it is
important to remember that while by law a man is guilty for violating the rules, in ethics a man is
guilty merely by \textit{considering} such violation. Be careful, be wise, and Baba Booey to all.

\part{The Dealer Circuit}

\chapter*{Introduction}
\addcontentsline{toc}{chapter}{Introduction}
As foreshadowed in the \hyperref[s:preamble]{pleasant preamble} above, having a loyal, dependable
dealer will be crucial to the reliable functionality of the crazy eights system. Without such a
dealer, the \textit{rules} of the game \textit{may not} be followed, causing complete and utter
pandemonium, ultimately resulting in a rather atrocious crazy eights game.\\\\
Therefore, it would benefit the system to create a robust \textit{finite state machine}, a machine
of a finite number of states. The system was reduced to a machine of 4 states (where $4 < \infty$
holds), of which include the state that waits for the previous request to end (denoted by A), the state
that waits for the next request to \textit{be} sent (denoted by B), the state that generates
random numbers, so as to deal the cards (denoted by C), and finally, the state that activates a
stack (or more accurately, a bi-directional Jenga tower) circuit (denoted by D). \\\\
All of the cleverness and power that has been teased above was encompassed by a circuit that has
since been named the \texttt{g07\_dealerFSM}. A more detailed insight to the design of the finite
state machine and the \texttt{g07\_dealerFSM} follows below, and is complemented by thoughtful
pictorial support for easier understanding.

\chapter*{Circuit Description}
\addcontentsline{toc}{chapter}{Circuit Description}

\begin{figure}[h]
	\begin{center}
		\caption{Pin-out diagram of the \texttt{g07\_dealerFSM} circuit}
		\includegraphics[scale=1.3]{dealer_symb}
	\end{center}
\end{figure}

Above is a pin-out diagram of the miraculous \texttt{g07\_dealerFSM} circuit, showcasing its input
and output ports. A more detailed description of these ports will be given below.

\section*{Description of ports}
\addcontentsline{toc}{section}{Description of ports}
\subsection*{request\_deal}
The \texttt{request\_deal} input bit is activated when the system requests that the dealer should
deal a card.
\subsection*{rand\_lt\_num}
The \texttt{rand\_lt\_num} input bit is controlled by an external circuit that is high when a random
number has a value that is less than the BJT's \texttt{NUM} output that the dealer is associated
with. This allows the dealer to tell whether it should keep generating random numbers (in state C),
or move on.
\subsection*{reset}
The \texttt{reset} input bit causes the finite state machine to be reset to its initial state when
\texttt{reset} is high. It is asynchronous.
\subsection*{stack\_enable}
The \texttt{stack\_enable} output bit is active when a valid random number has been generated in the
proper state. It is used to enable the BJT associated with the dealer to pop a card.
\subsection*{rand\_enable}
The \texttt{rand\_enable} output bit is active after \texttt{request\_deal} has been asserted anew,
and enables a random number generator to generate random numbers until \texttt{rand\_lt\_num} is
high.\\\\

A complete VHDL description of the \texttt{g07\_dealerFSM} circuit is provided in the following
section.

\section*{VHDL portrayal of the \texttt{g07\_dealerFSM} circuit}
\addcontentsline{toc}{section}{VHDL portrayal of the \texttt{g07\_dealerFSM} circuit}
\label{s:vhdl}
\lstinputlisting[language=VHDL,breaklines=true,frame=single]{g07_dealerFSM.vhd}

\chapter*{On the Testing of the Dealer Circuit}
\addcontentsline{toc}{chapter}{On the Testing of the Dealer Circuit}
It was an important and difficult challenge to resist being seduced by the beauty of the VHDL code
that describes the \texttt{g07\_dealerFSM} circuit. Despite it's beauty, a dealer cannot be trusted
without being tested. Think about this following section as an analog to an interview process, if
you will. Before putting the \texttt{g07\_dealerFSM} in commission, it was important to make sure it
satisfied all dealing requirements (and boy, did it ever), and to make sure it functioned correctly
on the hardware. The demonstrations that follow will prove that the \texttt{g07\_dealerFSM} is in
fact an excellent candidate.

\section*{The software circuit simulation: A poor man's analysis}
\addcontentsline{toc}{section}{The software circuit simulation: A poor man's analysis}
The first order of business in judging the effectiveness of the \texttt{g07\_dealerFSM} was to test
it with the magical Modelsim simulator. Unfortunately, said software was not functioning to the
tester's standards. As a consequence, the miraculous GtkWave VCD viewer came to the rescue, and the
\texttt{g07\_dealerFSM} was examined thoroughly according to a
\hyperref[a:dealertst]{carefully-written testbench}. \\For the less-enthused, the testbench starts the
state machine in its default state, and examines each state transition in the trivial order. Then,
the machine is brought to state C (where \texttt{rand\_enable} is high) and reset is activated to
ensure that it causes the machine to return to its initial state.\\
The resulting waveforms can be viewed below.\\\\
\includegraphics[scale=0.75]{dealertestcropped}
Of course, the results shown above cannot conclusively confirm the excellence of the
\texttt{g07\_dealerFSM} circuit on their own, because they give no evidence of the dealer
functioning on real hardware. Hence, this test was called a \textit{poor man's
analysis}. Fortunately, the testers \textit{are not} poor men. In fact, they have access to an
Altera DE1 FPGA development board equipped with an Altera Cyclone II FPGA. The tests in the
following section will describe how \textit{full confidence} of the \texttt{g07\_dealerFSM} was
obtained.

\section*{The hardware circuit simulation: an Engineer's victory}
\addcontentsline{toc}{section}{The hardware circuit simulation: an Engineer's victory}
Some tests are easy, leaving the participants in a relaxing state of mind. Yet, some tests are hard,
leaving participants in a state of panic. However, occasionally a test may reach a transcendental level of
difficulty, such that it may be said that the test \textit{separates the boys from the men}. That,
dear reader, is what you will witness in the remainder of this section. Anyone from laymen to
patricians can make a software circuit simulation, neglecting the possiblity that something may go
wrong on the hardware. While seemingly reasonable to the un-seasoned tester, this train of thought
is highly dangerous and irresponsible. It is up to the Engineer to realize that this, in fact, will
not suffice. This realization was exhibited, more than anything else, in the results below. \\\\
The first order of business was to create a \textit{test bed}, or circuit that bridges the gap
between human input and dealer communication. This was done using the not-short-of-incredible Altera
Quartus II schematic designer, and may be perused \hyperref[a:testbed]{here}.\\\\
For the less-motivated reader, a brief description of the testbed will be given. Firstly, The
\texttt{stack\_enable} output of the dealer circuit is passed as the \texttt{enable} input of the
BJT. The \texttt{rand\_enable} output was passed to the \texttt{enable} input of a
\texttt{g07\_register6} circuit, which is a masterfully-simple 6-bit register designed by the same
designers of the \texttt{g07\_dealerFSM}. The design of the \texttt{g07\_register6} is unfortunately
beyond the scope of this report. The register serves the purpose of latching a random
number generated by the random number generator module. Then, the value stored by the register is
passed through a comparator, which sends a 1 to the \texttt{g07\_dealerFSM} when the random value is
less than the \texttt{NUM} output of the BJT. The \texttt{request\_deal} input comes from the output
of the infamous \texttt{g07\_debounder} debouncer circuit (discussed in a previous laboratory
report), so as to remove the all-feared bouncing of the DE1's hardware buttons. The random number
generator module consists
of a \texttt{g07\_RANDU} (discussed in a previous laboratory report) which takes as input the
multiplexation of a constant seed, and the last generated random output of itself. When the BJT is
full, the constant seed is passed to the RANDU, and otherwise it uses its own seed to enliven
itself.\\\\
Finally, the \texttt{NUM} output of the BJT as well as the \texttt{VALUE} output corresponding to
the address described by the value stored in the \texttt{g07\_register6} are passed through
\texttt{mod13} circuits (discussed in previous laboratory reports) to be conveniently displayed on
the 7 segment decoders on the DE1 board.\\\\
Some say a picture is worth a thousand words, however only an infinite amount of words can be used
to describe the joy of the testers upon witnessing the massive success of the
\texttt{g07\_dealerFSM} circuit on the testbed. It would be infeasible to include so many pictures
in this report. Instead, a brief overview of what was accomplished will be offered.\\\\
There were just two more criteria that needed to be confirmed to ensure the proper functioning of
the dealer. Firstly, it needed to be confirmed that the \texttt{NUM} output of the BJT decreases by
only 1 after each deal (thus implying that the dealer deals one card at a time). This test passed
immediately with flying carpets. Furthermore, it had to be ensured that the cards being dealt were
random! Although this part did provide some inconsistencies early one (due to a mistake with the
comparator connections, mostly), eventually it was seen that random numbers in the range of 0 to
\texttt{NUM} were seen being popped from the BJT. Victory was gracefully achieved.

\begin{appendix}
	\chapter{VHDL Testbench for the \texttt{g07\_dealerFSM}}
	\label{a:dealertst}
	\lstinputlisting[language=VHDL]{dealerFSM_tst.vhd}
	\chapter{Testbed for the \texttt{g07\_dealerFSM}}
	\label{a:testbed}
	\includegraphics[scale=0.8,angle=0]{testbed_led}\\
	Above is the section of the testbed that showws how data is output to the 7 segment displays on
	the DE1. On the following page, the main body of the testbed may be observed.
	\includegraphics[scale=0.5,angle=270]{testbed_body}
\end{appendix}

\end{document}
